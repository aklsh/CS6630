\documentclass[12pt,a4paper,english]{paper}
\usepackage{fontspec}
\usepackage[T1]{fontenc}
\usepackage[utf8]{inputenc}
\usepackage[left=0.65in,right=0.65in,top=2cm,bottom=1in]{geometry}
\usepackage{multirow}
\usepackage{hyperref}
\usepackage{graphicx}
\usepackage{bm}
\usepackage[usenames,dvipsnames]{color}
\usepackage{booktabs}
\usepackage{fancyhdr}
\usepackage[most]{tcolorbox}
\usepackage{changepage}
\usepackage[square,sort,comma,numbers]{natbib}
\usepackage{amsmath}
\usepackage{amssymb}
\usepackage{eucal}
\usepackage[]{minted}
\usepackage{latexsym}
\usepackage{indentfirst}
\usepackage[ruled,vlined]{algorithm2e}
\usepackage[english]{babel}
\usepackage[autostyle, english = american]{csquotes}
\usepackage[default]{lato}
\usepackage{FiraMono}
\usepackage{lipsum}

\MakeOuterQuote{"}

\def \courseNumber {CS6600}
\def \courseName {Secure Processor Microarchitecture}
\def \assignmentName {Assignment 1}
\def \myName {Arjun Menon V, Akilesh Kannan}
\def \rollNumber {EE18B104, EE18B122}

\setlength{\headheight}{14pt}

\pagestyle{fancy}
\fancyhf{}
\rhead{\assignmentName}
\lhead{\courseNumber: \courseName}
\cfoot{\thepage}

% \linespread{1.2}

\definecolor{blue(ryb)}{rgb}{0.01, 0.28, 1.0}
\definecolor{green(ryb)}{rgb}{0.28, 1.0, 0.01}
\definecolor{red(ryb)}{rgb}{1.0, 0.01, 0.28}
\definecolor{black(ryb)}{rgb}{0, 0, 0}
\definecolor{gray(ryb)}{rgb}{0.75, 0.75, 0.75}
\definecolor{orange}{RGB}{255,155,0}
\definecolor{formalblue}{rgb}{0.95,0.95,1}
\definecolor{formalred}{rgb}{1,0.95,0.95}

\newenvironment{colorboxed}[4][gray]{
\begin{tcolorbox}[colback=#1!3!white,colframe=#1(ryb)!50!black,title=\textbf{#2 #3},#4]
}{
\end{tcolorbox}
}

\newenvironment{warning}{%
  \def\FrameCommand{%
    \hspace{1pt}%
    {\color{red}\vrule width 2pt}%
    {\color{formalred}\vrule width 4pt}%
    \colorbox{formalred}%
  }%
  \MakeFramed{\advance\hsize-\width\FrameRestore}%
  \noindent\hspace{-4.55pt}% disable indenting first paragraph
  \begin{adjustwidth}{7pt}{}%
  \vspace{2pt}\vspace{2pt}%
}
{%
  \vspace{2pt}\end{adjustwidth}\endMakeFramed%
}

\newenvironment{results}{%
  \def\FrameCommand{%
    \hspace{1pt}%
    {\color{blue}\vrule width 2pt}%
    {\color{formalblue}\vrule width 4pt}%
    \colorbox{formalblue}%
  }%
  \MakeFramed{\advance\hsize-\width\FrameRestore}%
  \noindent\hspace{-4.55pt}% disable indenting first paragraph
  \begin{adjustwidth}{7pt}{}%
  \vspace{2pt}\vspace{2pt}%
}
{%
  \vspace{2pt}\end{adjustwidth}\endMakeFramed%
}

\begin{document}
\thispagestyle{empty}
\vspace{-4.5cm}

\hspace*{-\parindent}
\begin{minipage}{0.65\textwidth}
\fontsize{22pt}{10pt}\selectfont\textbf{\assignmentName}\\[1mm]
\Large
\textit{\courseNumber: \courseName}\\[5mm]
\Large \myName \\[1mm]
\normalsize \rollNumber \\
\end{minipage}\hfill% push everything to the right
\raisebox{-13mm}{\includegraphics[scale=.28]{logo.pdf}}

\hrule \hrule
\medskip

\section{CLEFIA}
\section{Hashed Password Cracking}
\subsection{Approach}
The following observations were made about the hash implementation:
\begin{itemize}
    \item The hash only used the first \texttt{PRIME} characters of the input password. The rest of the characters didn't matter. This allows one to perform collision attacks on the hash, if the secret key was more than \texttt{PRIME} characters long - all the attacker has to do is to find the first \texttt{PRIME} characters in the secret key.
    \item Each character in the hashed string was dependent only on a single character from the input string. This allows one to easily reverse-engineer the hash by sequentially brute-forcing each of the characters in the correct order as they were used by the hashing mechanism. This reduced the number of tries exponentially - from \texttt{PRIME}$^\text{N}$ to \texttt{PRIME}$*\text{N}$, where N is the number of characters in the search space.
\end{itemize}

Based on these observations, we wrote a python script that starts with a know password (the blank password), and sequentially try brute-forcing the characters in the correct order as used by the hash. The correct character can be identified by a spike in the time taken for the hash computation (will increase by 1 sec). The next iteration of the cracker will use this character and try guessing the next character.

In order to automate this process, we used TCP sockets to communicate with the server.

\begin{colorboxed}{Password cracker script}{}{breakable}
    \inputminted[baselinestretch=0.85,breaklines,fontsize=\footnotesize]{python}{HashCollision/cracker.py}
\end{colorboxed}
\subsection{Results}

\begin{table}[H]
    \centering
    \begin{tabular}{|c|c|}
        \hline
         \textbf{Parameter}    &        \textbf{Value}                   \\ \hline
                {Password}     & \texttt{spm\{fastandfurious\}}          \\ \hline
                {Guesses}      &            153                          \\ \hline
    \end{tabular}
    \caption{Summary Table}
\end{table}

\begin{figure}[H]
    \centering
    \includegraphics[scale=0.3]{Q2_output.png}
    \label{fig:output_password}
    \caption{Password cracked and access granted}
\end{figure}

%Beginning References. Don't add any text beyond this.
%------------------------------------------

%\newpage %sending References to the last page

\bibliography{paper}
\bibliographystyle{acm}
\end{document}
